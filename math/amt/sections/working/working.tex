\documentclass[../main.tex]{subfiles}

\begin{document}
\chapter{Probability Theory}
\section{Binomial Distribution}
$$(x_{1}+x_{2})^{n}=\sum_{k=0}^{n}{n \choose k}x_{1}^{k}x_{2}^{n-k}$$
\section{Basic Probability}
A fiar 6-sided die is rolled 5 times. What is the probability of exactly two 3's?
\subsection{Outcomes}
Divide favorable outcomes by possible outcomes.
$$=\frac{{5 \choose 2}\cdot 5^{3}}{6^{5}}$$
\subsection{Raw Probability}
Find the probability of getting a favorable outcome.
$${5\choose 2}\left(\frac{1}{6}\right)^{2}\left(\frac{5}{6}\right)^{3}$$

\section{Basics of Expected Value}
$$\text{expected value}=\sum_{\text{results}}(\text{value})(\text{probability})$$
For a die,
$$\langle k\rangle=\frac{1}{6}(1+2+3+4+5+6)=\frac{6\cdot 7}{6\cdot 2}=3.5$$
$$\langle k^{2}\rangle=\frac{1}{6}(1^{2}+2^{2}+\cdots+6^{2})=\frac{6\cdot 7\cdot 13}{6}frac{1}{6}=\frac{91}{6}\neq \langle k\rangle^{2}$$
\subsection{Expectation of Square vs Square of the Expectation}
Consider,
$$\langle (k-\langle k \rangle)^{2}\rangle = \langle k^{2}-2k\langle k \rangle+\langle k \rangle^{2}\rangle=\langle k^{2}\rangle-\langle 2k\langle k\rangle\rangle+\langle \langle k\rangle^{2}\rangle$$
$$\langle k^{2} \rangle-\langle k\rangle^{2}$$
Since the LHS is $\geq 0$, this value is $>0$ so $\langle k^{2} \rangle > \langle k\rangle^{2}$.
The variance is equal to this LHS value: $\sigma_{k}^{2}=\langle (k-\langle k \rangle)^{2}\rangle$.
So, for the die, $\sigma_{k}=\sqrt{\frac{91}{6}-\frac{49}{4}}$. The probability of being in a standard deviation of a expected value is
$P(2\leq k\leq 5)=\frac{2}{3}$. If this distribution was normal, this value would be $\approx 68.2\%$.

\subsection{Independence and Products}
Given that $k_{1}$ and $k_{2}$ are two independent measurements, determine $\expect{k_{1}+k_{2}}$ and $\sigma_{k_{1}+k_{2}}^{2}$.
$$\expect{k_1 + k_2}=\expect{k_1}+\expect{k_2}$$
$$\sigma_{k_1 + k_{2}}^{2}=\expect{(k_1 + k_2)^{2}}-\expect{k_1 + k_2}^{2}$$
$$=\expect{k_{1}^{2}+2k_{1}k_{2}+k_{2}^{2}}-(\expect{k_{1}}^{2}+2\expect{k_{1}}\expect{k_{2}}+\expect{k_{2}}^{2})$$
$$=\expect{k_{1}^{2}}-\expect{k_{1}}^{2}+\expect{2k_{1}k_{2}}-2\expect{k_{1}}\expect{k_{2}}+\expect{k_{2}^{2}}-\expect{k_{2}}^{2}$$
Two outcomes are independent if and only if $\expect{k_{1}k_{2}}=\expect{k_{1}}\expect{k_{2}}$ always. So, given independence, we can simplify
$$\sigma_{k_1 + k_{2}}^{2}=\sigma_{k_1}^{2}+\sigma_{k_2}^{2}$$
The value $\expect{k_{1}k_{2}}-\expect{k_{1}}\expect{k_{2}}$ measures the correlation between $k_1$ and $k_2$.

\section{Multinomial Distribution}
This can be used to model distributions with more than 2 objects. Consider $n$ objects being placed in $m$ boxes.
The number of ways to place $r_{1}$ in box 1, $r_{2}$ in box 2, $\cdots$, and $r_{m}$ in box $m$ is
$${n \choose r_{1}r_{2}\cdots r_{m}}=\frac{n!}{r_{1}!r_{2}!\cdots r_{m}!};\sum_{i=1}^{m}r_{i}=n$$
Representing the distribution,
$$(x_{1}+x_{2}+\cdots+x_{m})^{n}=\sum_{r_{1}+r_{2}+\cdots r_{m}=n}{n \choose r_{1}r_{2}\cdots r_{m}}x_{1}^{r_{1}}x_{2}^{r_{2}}\cdots x_{m}^{r_{m}}$$
\subsection{Application}
A fair 6-sided die is rolled four times. $k_{1}$ is the number of 3's and $k_{2}$ is the number of 5's.
$$\expect{k_{1}}=\sum_{k=0}^{4}{4\choose k}k\left(\frac{1}{6}\right)^{k}\left(\frac{5}{6}\right)^{4-k}$$
Taking a derivative of the binomial expansion and multiplying by $x_{1}$,
$$x_{1}\frac{\partial }{\partial x_{1}}(x_{1}+x_{2})^{n}=nx_{1}(x_{1}+x_{2})^{n-1}=\sum_{k=0}^{n}{n\choose k}kx_{1}^{k}x_{2}^{n-k}$$
Applying this,
$$\expect{k_{1}}=4\cdot\frac{1}{6}=\frac{2}{3}$$

\section{Experimentation}
A certain quantity is measured $n$ times with the results $k_1 , k_2 , \cdots k_n$.
Asume the expected value of $k$ is $\bar{k}$ (unknown) and its standard deviation in $\sigma_{k}$ (unknown).
$$k_{\text{mean}}=\frac{1}{n}\sum_{i=1}^{n}k_{i}$$
Note that $k_{\text{mean}}\neq \bar{k}$. However,
$$\expect{k_{\text{mean}}}=\frac{1}{n}\sum_{i=1}^{n}\expect{k_{i}}=\frac{1}{n}\sum_{i=1}^{n}\bar{k}=\bar{k}$$
Note that the expected value of both $k$ and $k_{\text{mean}}$ is $\bar{k}$. So, let's analyze the standard deviation,
$$\sigma_{k_1 + k_2 + \cdots + k_n}^{2}=\sum_{i=1}^{n}\sigma_{i}^{2}=n\sigma_{k}^{2}
\Rightarrow \sigma_{\sum}=\sqrt{n}\sigma{k}
\Rightarrow \sigma_{\text{mean}}=\sqrt{n}\frac{\sigma_{k}}{n}=\frac{\sigma_{k}}{\sqrt{n}}
$$
Thus, taking the mean keeps the same expected value but divides the std. dev. by $\sqrt{n}$.
Note that we don't know the values of $\bar{k}$ and $\sigma_{k}$. So, let's calculate $\sigma_{k_{\text{mean}}}$.
$$\expectl{\sum_{i=1}^{n}(k_{i}-k_{\text{mean}})^{2}}=\sum_{i=1}^{n}\expect{(k_{i}-k_{\text{mean}})^{2}}=n\expect{(k_{1}-k_{\text{mean}})^{2}}$$
$$=n\expect{
    (k_{1}-\bar{k})^{2}
    -2(k_{1}-\bar{k})(k_{\text{mean}}-\bar{k})
    +(k_{\text{mean}}-\bar{k})^{2}
    }$$
$$=n\left[\sigma_{k}^{2}+\frac{\sigma_{k}^{2}}{n}-2\expect{(k_{1}-\bar{k})(k_{\text{mean}-\bar{k}})}\right]$$
Since $k_{1}$ and $k_{\text{mean}}$ are dependent, let's look at $k_{2}$.
$$\expect{(k_{1}-\bar{k})(k_{\text{mean}}-\bar{k})}=\frac{\sigma_{k}^{2}}{n}$$
Plugging this in,
$$\expectl{\sum_{i=1}^{n}(k_{i}-k_{\text{mean}})^{2}}=
n\left[\sigma_{k}^{2}+\frac{\sigma_{k}^{2}}{n}-2\frac{\sigma_{k}^{2}}{n}\right]=(n-1)\sigma_{k}^{2}$$
So,
$$\expectl{\frac{1}{n-1}\sum_{i=1}^{n}(k_{i}-k_{\text{mean}})^{2}}$$

\section{Large $n$}
Suppose we roll a fair six-sided die 6000 times. What is the probability a $2$ comes up between $990$ and $1050$ times?
$$P=\sum_{k=990}^{1050}{6000 \choose k}\left(\frac{1}{6}\right)^{k}\left(\frac{5}{6}\right)^{6000-k}$$
This is computationally intensive, so we can approximate this instead with an integral. Generalizing, say there are $n$ rolls and a $p$
probability. Using Sterling's approximation, $\ln n! \approx n\ln n- n + \frac{1}{2}\ln(2\pi n)$,

$$\ln {n\choose k}p^{k}(1-p)^{n-k}=\ln n! - \ln k! - \ln (n-k)! + k\ln p + (n-k)\ln (1-p)$$
$$\approx n\ln n - n+ \frac{1}{2}\ln (2\pi n)-k\ln k+k-\frac{1}{2}\ln(2\pi k)-(n-k)\ln(n-k)$$
$$+n-k-\frac{1}{2}\ln(2\pi (n-k))+k\ln p + (n-k)\ln (1-p)$$
We want to look at this for large $n$. Let $k=xn$.
$$\ln P_{k}\approx n\ln n - xn\ln(xn)-n(1-x)\ln(n-xn)+\frac{1}{2}\ln \frac{n}{2\pi xn^{2}(1-x)}$$
$$+xn\ln p + n(1-x)\ln(1-p)$$
$$=n\ln n - xn\ln n - xn\ln x - n(1-x)\ln n - n(1-x)\ln(1-x)$$
$$+\frac{1}{2}\ln\frac{1}{2\pi x(1-x)}-\frac{1}{2}\ln n+xn\ln p+n(1-x)\ln(1-p)$$
Cancelling and rearranging,
$$=n\left[x\ln p -x\ln x+(1-x)\ln(1-p)-(1-x)\ln (1-x)\right]+\frac{1}{2}\ln\frac{1}{2\pi n x(1-x)}$$
$$=n\left[x\ln\frac{p}{x}+(1-x)\ln \frac{1-p}{1-x}\right]+\frac{1}{2}\ln\frac{1}{2\pi n x(1-x)}$$
Similar to asymptotic expansions, let's look at the maximum. For large $n$, the last term is negligible.
Note that when $x=p$, the derivative and this expression vanish. 
$$\frac{\partial^{2}}{\partial x^{2}}\Rightarrow -\frac{1}{x}-\frac{1}{1-x}=\frac{-1}{x(1-x)}$$
So,
$$\ln P_{k}\approx -\frac{n}{2}\frac{(x-p)^{2}}{p(1-p)}+\frac{1}{2}\ln\frac{1}{2\pi n p(1-p)}$$
Substituting back to $k$,
$$=-\frac{1}{2n}\frac{(k-pn)^{2}}{p(1-p)}+\frac{1}{2}\ln\frac{1}{2\pi n (p)(1-p)}$$
Remember that $\sigma_{k}=np(1-p)$ from the binomial distribution:
$$P_{k}\approx \frac{\exp\left[-\frac{(k-np)^{2}}{2\sigma_{k}^{2}}\right]}{\sqrt{2\pi \sigma_{k}^{2}}}$$
This is the bell curve and is valid for large $n$. We also note that $np=\expect{k}$
Rewriting this, we get
$$\approx \frac{1}{\sigma_{k}\sqrt{2\pi}}e^{-\frac{1}{2}\cdot \left(\frac{k-\expect{k}}{\sigma_{k}}\right)^{2}}$$
So, back to our example,
$$P=\sum_{k=990}^{1050}{6000 \choose k}\left(\frac{1}{6}\right)^{k}\left(\frac{5}{6}\right)^{6000-k}
\approx\int_{989.5}^{1050.5}\frac{1}{\sigma_{k}\sqrt{2\pi}}e^{-\frac{1}{2}\cdot \left(\frac{k-1000}{\sigma_{k}}\right)^{2}}dk$$
Since numerical integrations don't work that well with large values, we can substitute to rescale with $u=\frac{k-\expect{k}}{\sigma_{k}};du=\frac{dk}{\sigma_{k}}$.
Recall that this $u$ is the $z^{*}$ score from statistics.
$$=\int_{z_{min}}^{z_{max}}\frac{e^{-\frac{u^{2}}{2}}}{\sqrt{2\pi}}du$$

\section{Birthday Problem}
There are $n$ problem in a room with random birthdays (none born on Feb. 29). How large must $n$ be
in order that the probability that at least two share the same birthday exceeds $\frac{1}{2}$.

\subsection{Solution}
Suppose we choose some fixed birthdays and then assign them:
$$P=1-{365\choose n}n!\left(\frac{1}{365}\right)^{n}$$
From a multinomial perspective, the probability of all different days is
$${365\choose n,365-n}{n\choose 1,1,\cdots,1}\left(\frac{1}{365}\right)^{n}$$
This generalizes well. Consider the case for one pair,
$${365 \choose 1,n-2,366-n}{n\choose 2,1,1\cdots 1}\left(\frac{1}{365}\right)^{n}$$
For two pairs,
$${365\choose 2,n-4, 367-n}{n\choose 2,2,1,1\cdots 1}\left(\frac{1}{365}\right)^{n}$$


\end{document}

\documentclass[../main.tex]{subfiles}
\begin{document}

\chapter{Convergence}
\section{Cauchy's Convergence Criterion}
The sequence $\{a_{k}\}_{k=0}^{\infty}$ of complex numbers converges
iff given any $\epsilon > 0\,\exists\, N\mid |a_{m}-a_{n}|<\epsilon$
whenever $m,n>N$. If $\{a_{k}\}_{k=0}^{\infty}$ converges to $L$,
then $|a_{m}-a_{n}|\leq |a_{m}-L|+|a_{n}-L|$

\section{Limit Superior}
$\limsup |a_{k}|=L$ means given any $\epsilon > 0$, $|a_{k}|>L-\epsilon$
for infinitely many $k$, but $|a_{k}|<L+\epsilon$ for only finitely many $k$.

\section{Laurent Series}
The Taylor series is guaranteed to converge inside the disk of convergence and 
is guaranteed to converge outside the disk of convergence. Convergence on the bondary
of the disk is unpredictable. Laurent series are similar to taylor series but negative 
power terms are allowed. 
$$f(z)=\sum_{k=-\infty}^{\infty}c_{k}(z-z_{0})^{k}:
c_{k}=\frac{1}{2\pi i}\oint_{C}\frac{f(z)}{(z-z_{0})^{n+1}}$$
Laurent series converge on an annulus: $r<|z-z_{0}|<R$. There can be multiple annuli of convergence
with the same center. There is a Laurent series that converges on the disk of convergence of the Taylor
series but without the center (punctured). The negative power terms limit the small circle and the
positive power terms limit the outer radius of convergence. Note that by FTCI,
$$\oint_{C}f(z)dz=2\pi i\cdot c_{-1}$$ 
This is because $c_{-1}$ is the coefficient of $\frac{1}{z}$.
If a Laurent series converging in a punctured disk about $z_{0}$ has an infinite number of negative power
terms, $0$ is an essential singularity of the function.

\section{Types of Singularities}
$z_{0}$ is an isolated singularity of $f(z)$ if $f$ is not analytic at $z_{0}$ but 
$\exists\, R:f(z)$ is analytic on $0<|z-z_{0}|<R$. It is a pointat which the function 
is not analytic (singular), but the funtion is analytic on a punctured disk centered at
that point. There are 3 types. Type I is if $|f(z)|$ is bounded on $0<|z-z_{0}|<R$. This is
a removable singularity. Type II is if $|f(z)|$ is not bounded but $|\frac{1}{f(z)}|$ is bounded.
Type II singularities can be removed by multiplying by $(z-z_{0})^{m}$ for some $m$. These are
known as poles of order $m$. If $m=1$, it is called a simple pole. After multiplying, $f(z_{0})\neq 0$
otherwise the $\frac{1}{|f(z)|}$ does not exist at $z_{0}$. Type III is neither type I nor II
and is known as an essential singularity. 

\section{Weirstrass Theorem}
This theorem states that if a function has an essential singularity, it gets arbitrarily close to
every complex number infintely many times on every punctured disk centered at the singularity.

\section{Picard's Little Theorem}
If $f$ is entire and non-constant, f takes on all values on the complex plane with the exception of 
at most 1 point.

\section{Picard's Great Theorem}
If $f$ is analytic and has an essential singularity at some point $w$. Then, on any punctured disk
centered at $w$, $f(z)$ takes on all possible complex values with at most a single exception infinitely
many times. This exceptional value is known as Picard's Exceptional Value.

\section{Equality and Convergence}
If there is a sequence of distinct points converging to $z_{0}$ at which $f(z)$ and $g(z)$ are both
analytic and $f(z_{k})=g(z_{k})$, then the Taylor series centered at $z_{0}$ for these two functions
is the same. By analytic continuation, for diferent series, this can be done for much of the complex plane.
Note that the Taylor series has no knowledge of branch cuts. Through continutation around a singular point,
Taylor series can give different values of $f$. From this, we can determine if there is a singular point.

\section{Residue}
If $z_{0}$ is an isolated singularity of $f(z)$, we define the \textbf{residue} of $f$ at $z_{0}$
as $$\text{Res}[f;z_{0}]=\frac{1}{2\pi i}\oint_{|z-z_0|=\epsilon}f(z)dz=C_{-1}$$
for small values of $\epsilon$ where $C_{-1}$ is the coefficient of $(z-z_{0})^{-1}$
in the Laurent Series converging to $f(z)$ on the punctured disk $0<|z-z_{0}|<R$.

\section{Residue Formula}
Suppose $z_{0}$ is a pole of order $m$ of $f(z)$. Then,
$$f(z)=\frac{c_{-m}}{(z-z_{0})^{m}}+\cdots+\frac{c_{-1}}{z-z_0}+\cdots$$
Making this function analytic at $z_{0}$,
$$(z-z_0)^{m}f(z)=c_{-m}+c_{-m+1}(z-z_{0})+\cdots+c_{-1}(z-z_0)^{m-1}+\cdots$$
Taking $m-1$ derivatives, we can set take the limit as $z\rightarrow z_{0}$ to isolate the $c_{-1}$ term.
So,
$$\lim_{z\rightarrow z_0}\left[\frac{d^{m-1}}{dz^{m-1}}(z-z_0)^{m}f(z)\right]=(m-1)!c_{-1}$$
Rearranging,
$$\boxed{\text{Res}[f;z_0]=c_{-1}=\frac{1}{(m-1)!}\lim_{z\rightarrow z_{0}}
\left[\frac{d^{m-1}}{dz^{m-1}}(z-z_0)^{m}f(z)\right]}$$
This is known as the \textbf{Residue Formula}.

\section{Using Residues}
Consider $\oint_{|z+2|=3}\frac{z^{3}e^{\frac{3}{z}}}{(z-2)^2}dz$. This has two singularities and the contour
only contains the essential singularity at $0$. We could use a Laurent Series with $R=2$ centered at $0$, but 
this may not be easy. Instead, we can consider the integral around all the singularities and subtract the residue
at $2$. (Let $I$ be the integrand)
$$=\oint_{|z|=5}Idz-2\pi i\text{Res}[I;2]$$
The large integral can be done with a laurent series centered at $0$ with $r=2$. 
$$\Rightarrow \frac{z^{3}}{z^{2}}\frac{e^{\frac{3}{z}}}{(1-\frac{2}{z})^{2}}$$
We need the $\frac{1}{z}$ term. After expanding with the binomial expansion,
$$\Rightarrow \cdots + \frac{1}{z}\cdot \frac{57}{2}+\cdots \Rightarrow 57\pi i$$
For the residue, we have a pole of order $2$. So,
$$\text{Res}[I;2]=\frac{1}{(2-1)!}\lim_{z\rightarrow 2}\frac{d}{dz}\left[
    (z-2)^{2}I
\right]=\lim_{z\rightarrow 2}\left[3z^{2}e^{\frac{3}{z}}
-z^{3}\cdot \frac{3}{z^{2}}e^{\frac{3}{z}}\right]
=6e^{\frac{3}{2}}$$
So, the original integral is
$$\therefore =\pi i(57-12e^{\frac{3}{2}})$$
So,
$$\text{Res}[I,0]=\frac{57}{2}-6e^{\frac{3}{2}}$$
This gives us a way to find the residue at an essential singularity.

\section{Real Integrals}
Consider the $\arctan$ integral $I=\int_{-\infty}^{\infty}\frac{dx}{x^{2}+9}$.
$$I=\lim_{R\rightarrow\infty}\int_{-R}^{R}\frac{dx}{x^{2}+9}$$
Closing the loop with an arc from $R$ to $-R$ such that $3i$ is inside, this loop $C$ has integral
$$\oint_{C}\frac{dz}{z^{2}+9}=2\pi i \text{Res}[\frac{1}{z^{2}+9},3i]$$
Since this is a simple pole,
$$=2\pi i\lim_{z\rightarrow 3i}\frac{z-3i}{z^{2}+9}=\frac{\pi}{3}$$
Calling the general integral in the limit $I(R)$,
$$\oint_{C}\frac{dz}{z^{2}+9}=I(R)+\int_{0}^{\pi}\frac{iRe^{it}dt}{(Re^{it})^{2}+9}$$
Using integral bounds, we can bound this part by:
$$\frac{R}{R^{2}-\frac{R^{2}}{2}}\cdot \pi \rightarrow 0$$
So, the original integral is simply $\frac{\pi}{3}$.

\subsection{Multiple Contributions}
Consider $\int_{-\infty}^{\infty}\frac{x^{2}dx}{(x^{2}+1)(x^{2}+4)}$.
Adding a semicircle, the integral of the closed loop is (let integrand be $\xi$)
$$\oint_{C}\xi dz=2\pi i\left(\text{Res}[\xi, i]+\text{Res}[\xi, 2i]\right)$$
Using L'Hopitals,
$$=2\pi i\left(\frac{i}{6}-\frac{i}{3}\right)=\frac{\pi}{3}$$
This loop integral is also,
$$=I(R)+\int_{0}^{\pi}\frac{(Re^{it})^{2}iRe^{it}dt}{(R^{2}e^{2it}+1)(R^{2}e^{2it}+4)}$$
Using integral bounds, we see the integral part goes to 0. So,
$$I=\frac{\pi}{3}$$

\subsection{Arc at Infinity}
Consider $P\int_{-\infty}^{\infty}\frac{(2x+5)dx}{x^{2}+2x+10}$. This integral diverges as a $\log$, but if we 
take it symmetrically, we can get a defined value. This is known as the \textbf{Cauchy Principle Value}.
Using L'Hopitals and residues, the integral of the closed loop (complete with semicircle) is
$$=(1+2i)\pi$$
This is also
$$=I(R)+\int_{0}^{\infty}\frac{(2Re^{it}+5)iRe^{it}dt}{(Re^{it}+1)^{2}+9}$$
As $R\rightarrow \infty$,
$$=I(R)+\pi\cdot 2i\Rightarrow I=\pi$$

\subsection{Auxiliary Parameters}
Consider $\int_{-\infty}^{\infty}\frac{dx}{(x^{2}+9)^{2}}$. We can consider instead
$\int_{-\infty}^{\infty}\frac{dx}{x^{2}+a}$. Using a semicircle, calculating the residue, and bounding 
the semicircle till it's muted, we see this $=\pi a^{-\frac{1}{2}}$.
We can take a derivative with respect to $a$ on both sides giving the value for the general case.
$$-\int_{-\infty}^{\infty}\frac{dx}{(x^{2}+a)^{2}}=-\frac{1}{2}\pi a^{-\frac{3}{2}}
\Rightarrow I=\frac{\pi}{54}$$

\subsection{Closing Below}
Previously, we only closed the loop on the upper-half of the plane. Regardless of which direction the loop is closed,
one the direction of the closed loop is accounted for ($-$ for clockwise), the value will be the same.

\subsection{Trignometric Real Integrals}
Consider $\int_{-\infty}^{\infty}\frac{\cos x}{x^{2}+1}dx$. If we try to close the loop, we notice that on the imaginary axis,
as the hemisphere grows, $\cos(iy)\rightarrow \infty$. This is a problem. Note that $\cos(x)=\Re(e^{ix})$.
So,
$$=\Re\left[\int_{-\infty}^{\infty}\frac{e^{ix}}{x^{2}+1}dx\right]$$
The integral over the close loop is
$$=2\pi i\cdot \frac{e^{ii}}{2i}=\frac{\pi}{e}=I(R)+
\int_{0}^{\pi}\frac{e^{iRe^{it}}}{R^{2}e^{2it}+1}iRe^{it}dt$$
Noticing $|e^{iRe^{it}}|=e^{-R\sin t}$, we bound the integral part,
$$<\left|\int_{0}^{\infty}\frac{R}{R^{2}-\frac{R^{2}}{2}}\cdot e^{-R\sin t}dt\right|$$
In the upper half of the plane, the exponential is $<1$. So,
$$<\frac{2\pi}{R}\rightarrow 0$$
So, the original integral $=\frac{\pi}{e}$ as well.

\section{Jordan's Lemma}
Let $p(z)$ and $q(z)$ be polynomials with $\text{degree}(q(z))\geq \text{degree}(p(z))+1$. Then,
$$\lim_{R\rightarrow\infty}\int_{|z|=R\text{ upper half}}\frac{p(z)}{q(z)}e^{iz}dz=0$$
In the worst case, the degree of $q(z)$ is exactly one greater than that of $p(z)$. Parameterizing
the upper semicircle, we can use integral bounds. Consider $p(z)=1,q(z)=z$. Then,
$$\left|\int_{0}^{\pi}\frac{e^{iRe^{it}}}{Re^{it}}iRe^{it}dt\right|
=\int_{0}^{\pi}e^{-R\sin t}dt=2\int_{0}^{\frac{\pi}{2}}e^{-R\sin t}dt$$
Graphing, we see $\sin t>\frac{2t}{\pi}$ (above the secant line). So,
$$<2\int_{0}^{\frac{\pi}{2}}e^{-R\cdot \frac{2t}{\pi}}dt=\frac{\pi}{R}\left[1-e^{-R}\right]
\rightarrow 0$$

\section{Harder Real Integrals}
\subsection{Long Trig Integrals}
Consider $\int_{-\infty}^{\infty}\frac{x\sin (2x)}{x^{2}-ix+2}dx$. Using the definition of $\sin(2x)$,
$$=\int_{-\infty}^{\infty}\frac{x\cdot \frac{e^{2ix}-e^{-2ix}}{2i}}{(x-2i)(x+i)}dx$$
Breaking apart,
$$=\frac{1}{2i}\left[\oint_{C+}\frac{ze^{2iz}}{(z-2i)(z+i)}dz
-\oint_{C-}\frac{ze^{-2iz}}{(z-2i)(z+i)}dz\right]$$
Here C+ is the upper semicircle and C- is the lower one. Using residues,
$$=\pi \frac{2ie^{-4}}{3i}+\pi\frac{-ie^{-2}}{-3i}=\frac{\pi}{3}(e^{-2}+2e^{-4})$$
Using Jordan's Lemma's Result and integral bounds, the semicircles have $0$ contribution
so the original integral is this as well.

\subsection{Keyhole Contour}
Consider $\int_{0}^{\infty}\frac{dx}{x^{2}+2x+2}$. A keyhole contour involves two full circles and introducing a
branch cut with an arbitrary power $\alpha$ in hopes to get an indeterminant form and use L'Hopitals to finally evaluate.

\subsection{Pizza Slice Contour}
Consider $\int_{0}^{\infty}\frac{dx}{x^{3}+1}=I$. Closing the curve with a "pizza slice" avoiding the other singularities,
$$\oint_{C}\frac{dz}{z^{3}+1}=I-\int_{0}^{\infty}\frac{e^{i\frac{2\pi}{3}}dx}{(xe^{i\frac{2\pi}{3}})}
=I-e^{i\frac{2\pi}{3}}I=-2ie^{i\frac{\pi}{3}}\sin\frac{\pi}{3}I$$
Using residues,
$$=2\pi i\text{Res}[\frac{1}{z^{3}+1};e^{i\frac{\pi}{3}}]=2\pi i\cdot \frac{1}{3e^{i\frac{2\pi}{3}}}
=\frac{2\pi i}{3}e^{-i\frac{2\pi}{3}}$$
Equating and rearranging,
$$I=\frac{\pi}{3\sin\frac{\pi}{3}}=\frac{2\pi}{3\sqrt{3}}$$
Consider this generally for
$$\int_{0}^{\infty}\frac{x^{\alpha -1}dx}{x^{n}+1}=I(\alpha)$$
Here, the integral around the "pizza slice" around one singularity is, (circular portions are 0)
$$=I(\alpha)+\int_{R}^{\epsilon}\frac{(xe^{\frac{2i\pi}{n}})e^{i\frac{2\pi}{n}}dx}{x^{n}+1}
=I(\alpha)-e^{i\frac{2\alpha \pi}{n}}I(\alpha)
=e^{i\frac{\alpha \pi}{n}}(-2i\sin \frac{\alpha\pi}{n})I(\alpha)
$$
Looking at this from the perspective of residues,
$$=2\pi i\cdot \frac{(e^{i\frac{\pi}{n}})^{\alpha -1}}{n(e^{i\frac{\pi}{n}})^{n-1}}$$
Equating, cancelling, and rearranging,
$$\boxed{I(\alpha)=\frac{\pi}{n}\csc \frac{\alpha \pi}{n}}$$

\subsection{Pole on the Contour}
Consider $P\int_{-\infty}^{\infty}\frac{\sin x dx}{x^{2}-x-6}$. Notice this has two simple poles on the contour.
Call the integrand $\xi$. Then,
$$=\lim_{\epsilon\rightarrow 0^{+}}\lim_{\delta\rightarrow 0^{+}}
\int_{-\infty}^{-2-\epsilon}\xi dx+\int_{-2+\epsilon}^{3-\delta}\xi dx
+\int_{3+\delta}^{\infty}\xi dx$$
Here, we can use $\Im$ to deal with $\sin$ and close around each simple pole either above or below.
WLOG, if we close above, the closed loop integral is
$$=I-\frac{ie^{-2i}}{5}\int_{\pi}^{0}dt+\frac{ie^{3i}}{5}\int_{\pi}^{0}dt$$
So, the original integral is
$$\frac{\pi}{5}\left[\cos 3-\cos 2\right]$$
For a simple pole, the contribution halfway aroudn the pole is simply half the residue at that pole.

\end{document}

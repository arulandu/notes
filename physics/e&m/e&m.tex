\documentclass{article}
\usepackage[utf8]{inputenc}
\newcommand{\kg}{\text{ kg}}

\title{Notes on 8.02x MIT OCW: Electricity and Magnetism}
\author{Alvan Caleb Arulandu}
\date{June 2021}

\usepackage{amssymb, amsmath, amsfonts}
\usepackage{graphicx}

\usepackage{mathtools}
\DeclarePairedDelimiter\bra{\langle}{\rvert}
\DeclarePairedDelimiter\ket{\lvert}{\rangle}
\DeclarePairedDelimiterX\braket[2]{\langle}{\rangle}{#1 \delimsize\vert #2}
\newcommand{\expect}[1]{\left\langle #1 \right\rangle}

\begin{document}

\maketitle
\section{Introduction}
This document contains compiled notes from 8.02x MIT OCW taught by Dr. Walter Lewin. 
\section{Electric Charges and Forces}
\subsection{Atoms}
Atoms consist of a nucleus of protons and neutrons with a cloud of electrons.
$m_{p}=m_{n}=1.7\times 10^{-27}\kg$  

\subsection{Conservation of Charge}
There are two different "kinds" of electricity: positive charge and negative charge.
The sum of all charges is conserved in a system. Charge cannot be created or destroyed.

\subsection{Induction}
Induction is a process in which a substance polarizes in response to interaction with
a charged substance. Conducting substances polarize since opposite charge attract.
Non-conducting substances stil polarize even though they prevent the flow of electrons
because the atoms slightly polarize (electrons spend more time on one side). This causes
polarization of the whole substance (electrons "shift"). Friction can transfer electric charge.

\subsection{Coloumb's Law}
Consider two charges $q_1,q_2$ separated by distance $r$ along direction $\hat{r}_{12}$ (from 1 to 2).
Let the force of 1 on 2 be $\vec{F}_{1,2}$ and vice versa be $\vec{F}_{2,1}$. Physicist Coloumb found
$$\vec{F}_{1,2}=\frac{q_{1}q_{2}K}{r^{2}}\hat{r}_{1,2}$$
Here $K$ is a constant. Coloumb is the SI unit for charge. The charge of a proton 
is $q_{p^{+}}\approx 1.6\times 10^{-16}C$. $K\approx 9\times 10^{9} \frac{\text{N}\text{m}^{2}}{\text{C}^{2}}$.
For historical reasons, the constant $K$ is written:
$$K=\frac{1}{4\pi \epsilon_{0}}$$
$\epsilon_{0}$ is known as \textbf{the permitivity of free space}. From experimental data, 
charges follow the law of superposition (add for multiple charges). Comparing Newton's Law
and Coloumb's Law, we see that electric forces are much stronger than gravitational forces
(think about two protons). In the nucleus, the protons repel each other strongly by electric forces;
\textbf{nuclear forces} are what hold the nucleus together. Our immediate surroundings are dictated by electric forces,
gravity dictates forces on large scales (planetary), and nuclear forces dictate on small scales.

\subsection{Electroscope}
This instrument consists of a conducting rod and two pieces of aluminum foil. If a charged object touches the top,
the entire instrument becomes the entire charge and the pieces of foil repel.

\section{Electric Fields}
The symbol $E$ is used for electric field.
Electric field is simply the force over the charge. By convention, it is in the direction
of the force for a positive test charge.
$$\vec{E}_{p}=\frac{\vec{F}}{q}=\frac{QK}{r^{2}}\hat{r}
\left[\frac{\text{N}}{\text{C}}\right]$$
For multiple charges, the force on some charge is calculated by superposition.

\subsection{Field lines}
In a vector field, the force on some charge is given by the vector at that point.
For field lines, the force on the charge is given by the direction of the tangent.
If the field line density is high, the strength of the field is higher (magnitude).
Field lines are not trajectories.

\subsection{Dipoles}
A \textbf{dipole} is when two charges have opposite sign but same magnitue.
In the presence of an electric field, a dipole will rotate as the electric force
exhibits torque.

\section{Electric Flux}
Consider a surface with surface element $dA$, normal vector $\hat{n}$,
and electric field $\vec{E}$ piercing through that element. Then,
the electric flux is $$d\phi =\vec{E}\cdot \vec{n}dA=\vec{E}\cdot d\vec{A}
=E\cos \theta\cdot dA$$
The total flux through a closed surface is the integral of this.

\subsection{Gauss' Law}
Consider a positive charge inside a sphere of radius $R$. The normal of each element is 
in the direction of the electric field. So, using the surface area, $\phi=4\pi R^{2}E$.
Substituting Coloumb's Law, 
$$\phi=\frac{Q}{4\pi \epsilon_{0}R^{2}}\cdot 4\pi R^{2}=\frac{Q}{\epsilon_{0}}$$
Note this result holds for any charge inside a closed surface. This is known as \textbf{Guass Law}.
It states the following,
$$\phi=\oint_{S}\vec{E}\cdot d\vec{A}=\frac{\sum Q}{\epsilon_{0}}$$
In other words the flux through a closed surface is the sum of the charge inside the surface
divided by the permitivity of free space. 

\subsection{Shell Theorem}
Consider a shell with small thickness and radius $R$ with charge $Q$ uniformly distributed on the surface.
Consider a point $r$ away from the center and its electric field there. If $r<R$,
then, by Guass' Law, $4\pi r^{2}E=\frac{Q}{\epsilon_{0}}=0\Rightarrow E=0$. For $r>R$,
$4\pi r^{2}E=\frac{Q}{\epsilon_{0}}\Rightarrow E=\frac{Q}{4\pi r^{2}\epsilon_{0}}$.

\subsection{Planes}
Consider a large plane with charge density $\sigma$. Consider $E$ at a distance $d$ from the plane.
Using a cylinder and symmtery, we can find that $E=\frac{\sigma}{2\epsilon}$. 

\section{Electric Energy}
\subsection{Electrostatic Potential Energy}
The electrostatic potential energy is denoted by $U$.
$$U=U_{\infty}-\int_{\infty}^{R}\frac{q_{1}q_{2}}{4\pi\epsilon_{0}r^{2}}dr
=\frac{q_{1}q_{2}}{4\pi\epsilon_{0}R}$$
Remember that only differences in potential energy help with anything. The above 
result takes $U_{\infty}=0$.

\subsection{Electric Potential}
The electric potential, $V$, is negative times the work per unit charge to go from $\infty$
to the location of the charge, $V_{p}$ (work-kinetic energy theorem). Consider two charges separated by $R$
with target charge being $q$ (both +). Then,
$$V_{p}=\int_{P}^{\infty}\vec{E}\cdot d\vec{R}=\frac{Q}{4\pi\epsilon_{0}R}=\frac{U}{q}$$
This is in unit "Volts". Similarly to contours of equal $z$ in multivariable calculus, 
\textbf{equipotential surfaces} are surfaces of constant electric potential. Electric field 
lines are always perpendicular to equipotential surfaces. By the defintion of electric potential,
$$V_{B}-V_{A}=-\int_{A}^{B}\vec{E}\cdot d\vec{r}$$
So, the energy released from a charge moving from point $A$ to point $B$ is
$$\Delta E=q(V_{A}-V_{B})=KE_{B}-KE_{A}$$
If there are no currents inside a metal, it is an \textbf{equipotential}.

\subsection{Relating Electric Potential and Energy}
We see that the electric field is the derivative of the potential.
$$V=\frac{Q}{4\pi\epsilon_{0}r} \Rightarrow
\frac{dV}{dr}\hat{r}=-\frac{Q}{4\pi\epsilon_{0}r^{2}}=-\vec{E}$$
Rearranging, we see that $E=-\nabla V$.

\subsection{Faraday Cage}
Consider a hollow closed conductor with some thickness. If a charge is introduced onto the surface (outside) of the
conductor, it will distribute across the surface but not the inside (using Guass' Law). So,
even in the presence of an electric field, though the conductor may polarize, no charge will accumulate 
on the inside. This is known as electrostatic shielding. Additionally, for a charge inside a spherical shell,
the shell will polarize. It turns out that to acheive this, with the conductor being equipotential, the electric field
being $0$, the charge will evenly distribute on the surface when polarizing. So moving the charge inside will not 
influence the outside world.

\section{High-voltage Breakdown}
\subsection{Charge Disribution}
Consider a solid conductor $A$ conducted to another conductor $B$ with a conducting wire.
Note the wire is significant in length. Then, the system is equipotential. So,
$$V_{A}=\frac{Q_{A}}{4\pi\epsilon_{0}R_{A}}\approx V_{B}=\frac{Q_{B}}{4\pi\epsilon_{0}R_{B}}
\Rightarrow \frac{Q_{A}}{R_{A}}=\frac{Q_{B}}{R_{B}}$$
Notice that if $R_{B}=aR_{A}$ for $a > 0$, $Q_{B}=aQ_{A}$. Then, the surface charge density (divide by SA)
is $\sigma_{B}=\frac{\sigma_{A}}{a}$. So for a solid conductor, the portions with higher curvature will have
a higher surface charge density and thus electric field.

\subsection{Electric Breakdown}
In the presence of high electric field, electrons collide with nitrogen/oxygen in the air displacing electrons
and creating ions. This causes an "avalanche". When the ions become neutral again, they produce light which we see
as a "spark". This heats up the surrounding air causing a pressure difference resulting in the sound.
An \textbf{electron volt} is a unit of energy; an electron from rest, when moving across a potential
difference of $1\text{ V}$ will cause an increase of one electron volt in the kinetic energy of th electron.
It takes an electric field of $\approx 3\times 10^{6}\frac{\text{V}}{\text{m}}$ for this to happen.
Using this, you can calculate the maximum voltage / charge on a Vandergraph, for example. 

\subsection{Current}
Current is charge per unit time and determines what "kills you". Voltage doesn't kill you current does.
An Ampere, A, is a Coloumb/second. $I=\frac{Q}{t}$.

\subsection{Lightning}
The bottom of the cloud is negatively charged while the top is positive. The ground has positive charge. A step leader
is a buildup of charge that travels to the ground producing little light. As charge builds up, 
return strokes take the charge back. These return strokes produce the most light; one bolt of lightning is many step leaders / strokes
like this since the pathway of the previous strike leaves a favorable path for the next. The process stops 
as the electric field decreases lowering the potential. There are other forms of lightning as well.

\subsection{St-Elmo's Fire}
Carbon arc discharge is formed by continuous current. This can also happen during thunderstorms to grass, corona discharge.
These discharges form ozone which has a distinct smell. Friction causing charge made the hydrogen catch fire in the Hindenbirg Zepplin.

\section{Electric Field Energy}
Let's first calculate the electric field from one plate on a point $P$ a distance $z$ 
from the plate. Only the vertical direction remains.
$$E=\int_{0}^{\infty}dr\int_{0}^{2\pi}d\theta
\frac{r\sigma}{4\pi \epsilon_{0}\left(\sqrt{r^{2}+z^{2}}\right)^{2}}\cdot \frac{z}{\sqrt{r^{2}+z^{2}}}
=\frac{z\sigma}{4\pi \epsilon_{0}}\cdot 2\pi \int_{0}^{\infty}dr\cdot 
r(r^{2}+z^{2})^{-\frac{3}{2}}$$
Substituting $u=r^{2}$,
$$=\frac{z\sigma}{2\epsilon_{0}}\int_{0}^{\infty}\frac{du}{2}\cdot (u+z^{2})^{-\frac{3}{2}}
=\frac{z\sigma}{2\epsilon_{0}}\cdot \frac{1}{2}\cdot -2(u+z^{2})^{-\frac{1}{2}}\biggr\vert_{0}^{\infty}
=\frac{z\sigma}{2\epsilon_{0}}\cdot\frac{1}{z}=\frac{\sigma}{2\epsilon_{0}}$$
Consider two thin plates of opposite and equal charge and equal area. 
Move one of the plates away by a distance $x$. The work required to do this
is $W$. Notice that the plate has a layer of charge on its surface if it has a thickness.
The electric field from this is clearly $\frac{\sigma}{\epsilon_0}$ (summing field of two plates). However, $E=0$ inside
the plate since it conducts. So,
$$W=Fx=Q\frac{0+\frac{\sigma}{\epsilon_0}}{2}x
=\frac{1}{2}\sigma AEx\cdot \frac{\epsilon_{0}}{\epsilon_0}
=\frac{1}{2}\epsilon_{0}E^{2}Ax$$
Notice that volume $V=Ax$. So, the \textbf{field energy density} (work / volume),
is $\frac{1}{2}\epsilon_{0}E^{2}$. So,
$$U=\int_{\text{all of space}}\frac{1}{2}\epsilon_{0}E^{2}dV$$
where $V$ is volume. For parallel plates, we simply have $U=\frac{1}{2}\epsilon_{0}E^{2}Ah$.
Substituting $Q=A\sigma$ and $V=Eh$, $U=\frac{1}{2}QV$.

\subsection{Capacitance}
Capacitance is the charge over potential: $C=\frac{Q}{V}$ with SI unit Farad.
Notice that for two objects of opposite polarities, the capacitance of object $A$
depends on the presence of the other object. So, capacitance is redefined as charge 
over potential difference.
$$C=\frac{Q}{V}\text{ or } C=\frac{\text{charge}}{\text{potential difference}}$$
For parallel plates, $C=\frac{Q}{V}=\frac{\sigma A}{Ed}=\frac{A\epsilon_{0}}{d}$.
A capacitor can be thought of as a device that can store energy. Photo flashes charge energy 
in a capacitor and release it through a lightbulb. A \textbf{fuse} is something that breaks if the 
current is too high.

\section{Dielectrics}
Dielectrics are substances that polarize in the presence of an electric field.
Consider a plate capacitor with opposite and equal charge density $\sigma_{f}$.
Let $E_{f}$ be the field produced by this. If the power supply used to create the potential 
difference is removed and a dielectric is moved in, there is an opposite $\sigma_{i}$
and $E_{i}$ that is induced and has opposite orientation. So, $E_{net}=E_{f}-E_{i}$.
Substituting, $\sigma_{i}=b\sigma_{f}$, $E_{net}=E_{f}(1-b)=\frac{E_{f}}{\kappa}$. Here $\kappa$
is known as the \textbf{dielectric constant}. So, Gauss' Law simply becomes
$$\oint \vec{E}\cdot d\vec{A}=\frac{1}{\epsilon_{0}}\cdot\frac{\sum Q_{free}}{\kappa}$$
Additionally, the capacitance in the plate capactor becomes $C=\frac{A\epsilon_{0}}{d}\kappa$.
Some substances are naturally dipoles and have much higher $\kappa$ than induced dipoles.

\section{Electric Curents}
Convention states that current is in the direction of positive charge.

\subsection{Ohm's Law}
This is hard to derive without quantum mechanics, but we can try.
Consider copper at 300 K where the free electrons have an average speed of $\expect{v_{e}}$.
Let $\tau$ be the time between collisions of free electrons with atoms and let $n$ be the number
of free electrons per cubic meter. If a potential difference is applied, a force is applied to them.
$$F=eE;\, a=\frac{F}{m_e};\, v_{d}=a\tau =\frac{eE}{m_{e}}\tau$$
Here $v_d$ is the drift velocity or the velocity picked up by the electrons.
Now consider a wire with length $l$ and cross-sectional area $A$. Over one second,
$$I=(v_{d}A\cdot 1\text{ s})\cdot \frac{ne}{1\text{ s}}=\frac{e^{2}n\tau}{m_{e}}AE=\sigma AE$$
% QUESTION: Doesn't this not work by unit analysis?
Here, we call the constant in front $\sigma$ known as the \textbf{conductivity}. Using $E=\frac{V}{l}$,
$$V=\frac{l}{\sigma A}I=RI$$
This constant of proportionality $R$ is resistance. This is \textbf{Ohm's Law}.
Resistivity is $\rho=\frac{1}{\sigma}$. So, $R=\frac{l\rho}{A}$. Resistance is in units "Ohm's".
Resistance depends heavily on temperature which completely breaks Ohm's Law.

\subsection{Multiple Resistors}
Consider two resistors in series with potential difference $A$ across the entire wire and current $I$.
Since resistance is linear with length, $V=I(R_{1}+R_{2})$ and the voltage across each resistor is $V_{1}=IR_{1}$
and $V_{2}=IR_{2}$. For parallel resisitors, by Ohm's Law, we see that $V=I_{1}R_{1}=I_{2}R_{2}$. Notice 
$I=I_{1}+I_{2}$.

\section{Circuits}
We know $E=\frac{\Delta V}{\Delta x}$. So, it is in the direction of + to -. So, in a circuit,
the battery has current opposing the direction of electric field. The chemical energy in the battery
allows this to happen.

\subsection{Basic Battery}
In a copper-zinc battery, a plate of $\text{Cu}^+$ and $\text{Zn}^-$ are placed
in a $\text{H}_{2}\text{SO}_{4}$ solution. This solution disassociated when current runs through it and the
$\text{SO}_{4}^{-}$ ions carry the charge (negative) from copper to zinc plate. As this happens, copper precipitates
on the copper plate and some of the zinc dissolves. The reason why the charge carrier travels apposing the 
electric field is because it undergoes a chemical reaction that releases more energy than required to "go up the electric field".
Running current in the opposite direction reverses the process and "recharges" the battery.

\subsection{Varying Resistance}
If $R=\infty$, then $I=0$, so $V_{\text{over battery}}=V_{b}=\varepsilon$.
This $\varepsilon$ is known as the electromotive force. There is always some small resistance 
in the wire $r_{i}$. So, for some finite $R$, $\varepsilon = I(R+r_{i})$ and $V_{b}=IR=\varepsilon-r_{i}$. If there 
is no additional resistance (shorting the battery), $I_{max}=\frac{\varepsilon}{r_{i}}$ and $V_{b}=0$.

\subsection{Power}
Consider two points such that $V_{A}>V_{B}$ and moving some tiny charge $dq$ from $A$ to $B$.
The work that the electric field is doing is $dW=dq(V_{A}-V_{B})$. Dividing by $dt$,
$$P=I\Delta V$$
If Ohm's Law also holds, $P=\frac{V^{2}}{R}=I^{2}R$. Power is in units J/sec or Watts.
The body produces power which can be seen as infrared radiation / heat. So, $P_{max}=\frac{\varepsilon^{2}}{r_i}$.
This power is what causes the heat when shorting out a battery. Electric companies charge for energy not power.

\subsection{Kirchoff's Rules}
The first rule states that $\oint \vec{E}\cdot d\vec{l}=0$ (closed loops on conservative fields).
The second rule is conservation of charge.

\subsection{Kelvin Water Dropper}
One water source has two paths of travel through two open conducting cylinders and landing in two conducting trash cans 
that are connected to the opposite cylineder. A spark forms between the trashcan as the water spreads out causing it to come back together again.

\section{Magnetism}
Iron oxide is known as magnetite and has two poles at which maximal attraction occurs. 
Opposite poles attract and like poles repel. Magnetic monopoles do not exist (as far as we know).
The \textbf{magnetic field}, $\vec{B}$, goes in the clockwise direction for current going into the board (right-hand), by convention.
Current generates a magnetic field. Similarly, the magnetic field creates a force on the wire $\hat{F}=\hat{I}\times\hat{B}$.

\subsection{An Observation}
Consider two parallel wires with current flowing in the same direction. The current in one wire causes a magnetic field 
resulting in a force on the other wire. This causes the wires to be attracted. If one current changes direction,
this will cause a repulsion.

\subsection{Lorentz Force}
Since there are no magnetic monopoles, a force in the form $\vec{F}=q_{B}\vec{B}$ (as in electricity) can't be used.
Instead, we consider an electric charge $q$ moving with velocity $\vec{v}$. From experiment, the $\vec{F}\perp \vec{v}$,
$F_{B}\propto v$ and $F_{B}\propto q$. So, $$\vec{F_{b}}=q(\vec{v}\times\vec{B})$$
This is known as the \textbf{Lorentz Force}. The strength of the magnetic field is in Tesla's 
or $\frac{\text{N}\cdot \text{sec}}{\text{C}\cdot \text{m}}$. We also use the unit Gauss: $1G=10^{-4}T$
since $1T$ is extremely large for a magnetic field. So, the total force on a charge is
$$\vec{F}_{tot}=q\vec{E}+q(\vec{v}\times\vec{B})=q(\vec{E}+\vec{v}\times\vec{B})$$
Consider a positive charge $+dq$ moving along a current $I$ with drift velocity $v_{d}$ (positive for convention - doesn't matter).
$$d\vec{F}_{B}=dq(\vec{v_{d}}\times \vec{B})=Idt(\vec{v_{d}}\times \vec{B})$$
By kinematics, $d\vec{l}=\vec{v_{d}}dt$. So,
$$d\vec{F}_{B}=I(d\vec{l}\times \vec{B})$$
This means that the force at the location is the current times the distance the charge travels cross the
local magnetic field. Integrating through the wire gets the total Lorentz force.

\subsection{Electric Motors}
Consider a rectangular circuits with a constant $\vec{B}$ perpendicular to the widths of the rectangle.
Then, the force on one side will be $IaB$ and the force on the other side is $-IaB$ if $a$ is the width of the 
rectangle. This exhibits a torque of $\tau=2\cdot IaB\cdot \frac{l}{2}=IaBl$. As this rotates, 
the torque decreases to $0$ when the rectangle is rotated 90 degrees. Now, the net force is $0$.
A little inertia (nudge) causes the rectangle to rotate further till it hits 180 degrees before it reverses
and the process continues. Adding a spring between the terminals will create a \textbf{current meter}.
If we have two wires connected to a battery with a slipping contact (brushes), the wires won't twist and the current 
will automatically switch directions after 180 degree rotation (\textbf{commutator}) preventing torque reversal, allowing the motor to 
rotate in one direction only.

\section{Moving Charges}
Consider a constant $\vec{B}$ out of the page and a charge $q$ with velocity $\vec{v}$.
Then, the charge travels in a circle of radius $R$. By centirpetal motion,
$$|\vec{F}|=|q(\vec{v}\times\vec{B})|=qvB=m\frac{v^{2}}{R}\Rightarrow \boxed{R=\frac{\rho}{qB}}$$
Here $\rho$ is the momentum of the charged particle. Using kinetic energy, $KE=q\Delta V=\frac{1}{2}mv^{2}$.
We can express $R$ in terms of potential,
$$R=\frac{mv}{qB}=\frac{m}{qB}\cdot \sqrt{\frac{2qV}{m}}=\sqrt{\frac{2mV}{qB^{2}}}$$
However, for large energies, we can get a speed larger than the speed of light. 
If we make relativistic corrections, let the Lorentz factor be $\gamma$ and our results change like so.
$$\gamma=\frac{1}{\sqrt{1-\frac{v^2}{c^2}}};\,
R=\gamma\frac{\rho}{qB}=\sqrt{\frac{(\gamma +1)mV}{qB^{2}}};\,
KE=(\gamma-1)mc^{2}$$

\subsection{Mass Spectrometers}
Consider two isotopes of uranium $U^{92}_{146},U^{92}_{143}$. These can be separated using a mass spectrometer.
First, we heat the uranium so it ionizes and accelerate over it a potential difference. 
Using the radius, since the mass of the particles are different, we can separate the isotopes.

\subsection{Particle Accelerators}
A \textbf{cyclotron} consists of two chambers with a magnetic field. As an electron is released, it increases its energy 
due to the potential difference across each chamber. The difference is then switched so it can complete a revolution.
This increases the radius and the electron continues an outward
spiral path, gaining speed. This is used to accelerate charged particles to high speeds.
Note that the time for a revolution is $T=\frac{2\pi R}{v}=\frac{2\pi m}{qB}\gamma$ which is independent of the speed
of the particle for non-relativistic speeds. Using $T$, we can figure out the frequency to change the potential. \textbf{Synchrotrons}
or \textbf{synchocyclotrons} account for relativistic effects and don't have a set frequency.
Modern accelerators have rings of constant radius and increase the magnetic field to maintain the radius to keep the particle in the ring.
The Large Hadron Collider and others use superconducting magnet techniques to do this. As particles move through the air, they create ions that cause it to lose 
kinetic energy till it comes to a halt. \textbf{Cloud chambers} are used to see the paths of the particles in accelerators.
They consist of liquid alcohol that is cooled in the the bottom (undercooled) so that the alcohol freezes in drops around the ions when they are 
formed. This allows us to visualize paths, ions, and particles. The \textbf{bubble chamber} is an advanced cloud chamber that uses
liquid hydrogen which formes gas bubbles around the ions.

\section{Biot-Savart Formulation}
\subsection{Biot-Savart}
Consider a wire with current $I$. From experiment, if magnetite is put around a wire forming a circle of radius $R$,
$B\propto\frac{I}{R}$. Notice that for electricity, electric monopoles (one charge) give fields of $\frac{1}{r^{2}}$.
Integrating this gives the $\frac{1}{R}$ field. Similarly, consider the general magnetic field element for some length of
wire with current. The \textbf{Biot-Savart Formulation} states that ($\hat{r}$ is the radial direction)
$$d\vec{B}=\frac{CI}{r^{2}}(d\vec{l}\times\hat{r})$$
Here $C$ is a constant that is also seen as $C=\frac{\mu_{0}}{4\pi}$.
Here $\mu_0$ is known as the \textbf{permeability of free space}. For a wire, the magnetic field at point $P$, perpendicular
distance $R$ from the wire is:
$$B=\int d\vec{B}=\frac{\mu_{0}I}{4\pi}\int_{-\infty}^{\infty}\frac{1}{r^{2}}(d\vec{l}\times \hat{r})
=\frac{\mu_{0}I}{2\pi}\int_{-\infty}^{0}\frac{1}{r^{2}}\cdot \sin\theta dl$$
Evaluating this integral,
$$\int_{-\infty}^{0}\frac{1}{r^{2}}\cdot \sin\theta dl=\int_{-\infty}^{0}\frac{R}{(R^{2}+l^{2})^{\frac{3}{2}}}dl
=\frac{x}{R}\cdot\frac{1}{\sqrt{R^{2}+x^{2}}}\biggr\vert_{-\infty}^{0}
=\lim_{x\rightarrow\infty}\frac{x}{R\sqrt{R^{2}+x^{2}}}$$
$$=\lim_{x\rightarrow\infty}\frac{1}{R\sqrt{\frac{R^{2}}{x^{2}}+1}}=\frac{1}{R}$$
So, 
$$B=\frac{\mu_{0}I}{2\pi R}$$
For a circular wire,
$$B=\frac{\mu_{0}I}{4\pi R^{2}}\cdot\int (d\vec{l}\times\hat{r})
=\frac{\mu_{0}I}{4\pi}\cdot 2\pi R=\frac{\mu_{0}I}{2R}$$
Unless there is a magnetic monopole,
$$\boxed{\oint_{S}\vec{B}\cdot d\vec{A}=0 \Rightarrow \nabla B=0}$$
This is the second of Maxwell's four equations.

\subsection{High-voltage Power Lines}
Consider a long wire. Then, $V_{A}-V_{B}=IR\rightarrow V_{B}=V_{A}-IR \rightarrow V_{B}I=V_{A}I-I^{2}R$.
Notice that the power delivered from $A$ releases $I^{2}R$ in heat. This needs to be minimized. At constant
power, increasing the voltage decreases the current which greatly decreases this loss in energy delivered per unit time.
This is more cost effective than decreasing $R$ by using gold, expensive superconductors, or using thicker wires.
However, the voltage has to be at a reasonable level to prevent corona discharge (this is why lines glow during storms).

\subsection{Leyden Jar}
A Leyden jar consists of two conducting beakers with a nonconducting beaker in between. The conductors are charged 
and the dielectric gets an induced charge. After disassembling and taking the free charge off, the induced charge must go away,
but when shorting, there is a spark. This means that there must be free charge on the dielectric. Remember that there is 
air between the dielectric and the conductor. So, $2E_{air}d_{air}+\frac{E_{air}}{\kappa}d_{glass}$. If you calculate,
the electric field in the glass is higher than the corona discharge electric field causing charge to be stored on the glass.
So, when the jar is disassembled, the charge on the glass is what makes the spark after re-assembly.

\section{Ampere's Law}
Consider a current $I$ running into the page producing $\vec{B}$. We know that $B=\frac{\mu_{0}I}{2\pi R}$. 
Ampere's Law states that integrating $\vec{B}$ over any closed loop, $C$, perpendicular to the wire will be $\mu_{0}I$.
Consider the circular loop, then $\oint_{C}\vec{B}\cdot d\vec{r}=\frac{\mu_{0}I}{2\pi R}\cdot 2\pi R=\mu_{0}I$.
For any open surface attached to the curve,
$$\oint_{C}\vec{B}\cdot d\vec{l}=\mu_{0}I_{enc}$$
where $I_{enc}$ is the current enclosed by the surface (penetrating). Now, consider a uniform current through a wire with radius $R$.
Let $B$ be the magnetic field at $P$ a distance $r$ from the center line of the wire. Using Ampere's Law,
\[
B=
\begin{cases}
    \frac{\mu_{0}I}{2\pi r} & r>R \\
    \frac{\mu_{0}Ir}{2\pi R^{2}} & r<R \\ 
\end{cases}
\]

\subsection{Solenoids}
A solenoid is a spiraled wire (slinky). For a circular wire, we know the magnetic field is almost straight through the center.
The spiral allows for a "straight", constant magentic field inside the spiral and outside the solenoid is almost zero. 
Using a rectangle with one side through the center
of the loops along with Ampere's Law allows us to calculate the magnetic field. Note that all sides except the one through the loops
are $0$ in the integral. Let the solenoid be of length $L$ with $N$ loops. Then for a rectangle with length $l$,
$$Bl=\frac{l}{L}\cdot N\cdot I\mu_{0}\Rightarrow B=\frac{\mu_{0}IN}{L}\text{ for } L\gg R$$

\subsection{Kelvin Water Dropper}
Assume there is some small charge on one of the open cans. This will polarize the water source so that the opposite sign charge 
will fall in the drop. This makes the buckets collect like charges which reinforce the charge on the cans (since it's connected via a conductor).
So, causes even more polarity and builds the potential difference between the buckets until there is a discharge. 
Notice the current and electric field direction match at the top and bottom but not in the cans; this is where gravity is doing the work.
As the charge in the can builds, the negative drops spread out by attraction which is why, right before the spark, the water "frays".

\section{Electromagnetic Induction}
Faraday setup a solenoid with a second loop around the solenoid. As current ran through the circuit, the magnetic field was constant
and no current was in the second loop. But, when a switch was toggled, the magnetic field changed, causing a current; he concluded that
a changing magnetic field causes current. This is known as \textbf{electromagnetic induction}.

\subsection{Lenz Law}
Consider a bar magnet moving towards a square, closed wire. As it moves, the magnetic field increases, so a magnetic field 
from current in the wire is produced to oppose this increase. Moving the magnet up will reverse the current. The current wants
to oppose the change in magnetic field. Systems don't like change. This is useful for direction but not the magnitude.
From Ohm's Law, the induced electromagnetic force is the resistance times the induced current.
$$\varepsilon_{ind}=I_{ind}R$$

\subsection{Faraday's Law}
From experiment, Faraday found that the EMF produced by the second loop is proportional to the change in magnetic field
over time and the area enclosed by the second loop. $\varepsilon_{2}\propto \frac{dB}{dt}$ and $\propto \text{area}$.
This suggests that it is proportional to magnetic flux.
$$\varepsilon=-\frac{d}{dt}\phi_{B}=-\frac{d}{dt}\int_{S}\vec{B}\cdot d\vec{A}=\oint \vec{E}\cdot d\vec{l}$$
The minus is simply to signify the direction that opposes the change in flux (from Lenz Law). Together, with the minus sign,
these are known as the \textbf{Faraday-Lenz Law}. The direction of $d\vec{A}$ for pursists follows the right-hand-rule, but using
Lenz Law should work all the time. For changing magnetic fields, the electric fields are {\it non-conservative} which means Kirchoff's
Rule doesn't apply. Notice if $\frac{d\phi_{B}}{dt}=0$, Faraday's Law becomes Kirchoff's Law.

\subsection{Transformers}
Now, consider the second loop having multiple loops around the solenoid. Then, there is more flux which increases EMF.
This is the idea behind transformers: get "any" EMF by changing the number of loops around the solenoid.

\section{Motional EMF}
Consider a closed conducting loop (flat - dimensions $x,y$) with constant $\vec{B}$ making angle $\theta$ with the normal.
Then, $\phi_{B}=xyB\cos\theta$. To induce a current, either the area of the loop, $B$, or $\theta$ can change with time.

\subsection{AC Current}
Consider the closed loop rotating at angular frequency $\omega$. If $\theta=0$ at $t=0$, $\theta(t)=\omega t$.
So, $$I(t)=\frac{\varepsilon(t)}{R}=\frac{1}{R}\cdot -\frac{d\phi_{B}}{dt}
=\frac{1}{R}\cdot -\frac{d}{dt}\left[AB\cos (\omega t)\right]
=\frac{A}{R}\cdot B\omega \sin (\omega t)
$$
Notice that the current alternates direction due to the $\sin$. This is known as \textbf{AC current}. Notice 
that a closed loop with $N$ windings will have $N$ times as much EMF because $\vec{B}$ penetrates $N$ times more area.
However, if the loop is rotating about an axis parallel to $\vec{B}$, there will be no EMF since $\vec{B}\cdot d\vec{A}=0$
always. In the US, $f=60\text{Hz}$; regular lightbulbs still produce light when $\sin$ passes a $0$ but flourescent lights 
do not - hence, stroboscopes flicker.

\subsection{Dynamo}
A dynamo is a machine that produces AC current.
It can do so with a turbine that rotates conducting loops in a magnetic field. Increasing the number of windings, angular frequency,
loop area, and strength of the magnetic field will all increase EMF. Many objects are synched with the frequency like clocks.

\subsection{Changing Area}
Consider a loop with a crossbar that moves to increase the area (length $l$ and width $x$). Consider a vertical $\vec{B}$.
Then,
$$\varepsilon=-\frac{d\phi_{B}}{dt}=-\frac{d}{dt}lxB=-lBv$$
For the crossbar, the Lorentz force is $\vec{F}_{l}=l(\vec{I}\times\vec{B})$ and must be surpassed which requires work.

\subsection{Eddy Currents}
Consider a conductor moving into a magnetic field $\vec{B}$. As it moves in, $\phi_{B}$ increases, so a current is produced
along some path in the conductor. These are known as Eddy Currents. Moving windings through a magnetic field produces a current that can power things like lightbulbs.
Cutting into the conductor (ex. teeth) increases resistance and reduces Eddy Currents.

\subsection{Magnetic Braking}
These currents produce heat from the resistance which causes the loop to lose kinetic energy. This means that the loop slows down.
This can also seen using the Lorentz force which is in the opposite direction of the loop's movement toward the field source.

\section{Displacement Current}
Consider a current $I$ running to a capacitor with plate radius $R$. Then,
$$\frac{dE}{dt}=\frac{d}{dt}\left[\frac{\sigma_{free}}{\kappa\epsilon_0}\right]
=\frac{I}{\pi R^{2}\kappa \epsilon_0}$$
Now consider some point $P$ before the capacitor a distance $r$ from the wire. 
Using Ampere's Law on the flat surface, we get $B=\frac{\mu_{0}I}{2\pi r}$.
However, consider the net-like surface that paces between the capacitor plates. This surface
has no current piercing it so $B=0$. This inconsistency was remedied by Maxwell who added a 
term; since change in magnetic flux resulting in current (Faraday), he proposed that changing
electric flux resulted in magnetic field. Maxwell added the following term,
$$\oint_{C}\vec{B}\cdot d\vec{l}=\mu_{0}\left(I_{pierce}+\epsilon_{0}\kappa \frac{d\phi_{E}}{dt}\right)
\text{ where }
\phi_{E}=\oint_{S}\vec{E}\cdot d\vec{A}$$
This added term is known as \textbf{displacement current}. With this, consider $P$ inside the capacitor.
Then, (neglecting the fringe field)
$$B\cdot 2\pi r=\mu_{0}\epsilon_{0}\kappa \cdot \pi r^{2}\cdot \frac{I}{\pi R^{2}\kappa \epsilon_{0}}
\Rightarrow B=\frac{\mu_{0}Ir}{2\pi R^{2}}$$

\subsection{Synchronous Motors}
Consider a loop of wire, similar to the structure in a motor that is rotating with a constant magnetic field.
If we add two more loops of wire (identical) at 120 degree offsets, we produce a \textbf{three phase current} that acheive 
maximums at offset times. Consider this with solenoids that are all at offsets. When one reaches maximum EMF, the others are two times lower but their vectorial sum 
points in the same direction. As this continues, the magnetic field rotates a full 360 degrees in one period.
If we now place a magnet in the center of the solenoid ring, the magnet will rotate to follow the magnetic field.
This is called a \textbf{synchronous motor}. If you use a conducting sphere, the resulting Eddy Currents will cause a torque.
This does not use brushes and is called an \textbf{induction motor}.

\subsection{Magnetic Top}
The magnetic top spins and underneath is a box with a battery, solenoid, and transistor-controlled switch.
The field from the top creates an induced current which is sensed by the transistor and turns on the switch.
This creates a magnetic field in the solenoid (south pole) and causes the top to rotate a little bit. As the north
pole recededs / south pole approaches, the EMF reverses which causes the transistor to turn the switch off. This removes the south pole
and the top keeps spinning. This is a type of induction motor that produces a magnetic field for half the period.

\subsection{Two-phase Current}
Consider two solenoids at 90 degree offsets with one horizontal and another vertical. The net magnetic field 
rotates and placing a conductor creates another induction motor.

\section{Magnetic Applications}
\subsection{Human Heart}
The heart is divided into two chambers and each cell pumps in/out ions.
Pacemaker cells change their potential -80 to +20 millivolts. This causes the other cells to follow and a wave forms over the heart as it contracts.
As the wave crosses the cell, its electrical field takes the shape of a dipole in a process known as \textbf{depolarization}.
The cell then \textbf{repolarizes} from the bottom back to -80 mV. This causes an electric field and a potential difference 
between different parts of your body which allows the heart to pump blood. Synchronization is key to proper heart function.

\subsection{Aurora Borealis}
Charged particles spiral and follow the curved magnetic field lines to earch. The sun emits plasms (highly ionized electrons and protons)
as a solar wind. This ionizes the upper atmosphere of the earth which produces the Aurora. The color depends on energy, height, and ionization
or oxygen/nitrogen ionization.

\subsection{Superconductivity}
If you cool mercury with liquid helium, mercury loses all of its resistivity. Superconducting coils can create strong magnetic fields
but superconductors cannot have electric field inside of it which means current approaches infinity. The \textbf{magnetic pressure}
is $P=\frac{B^{2}}{2\mu_{0}}$. The eddy currents are created to make the magnetic fields inside the superconductor $0$.
This can allow for \textbf{magnetic levitation}.

\subsection{Magnetic Levitation}
Consider a magnet moving over a conducting plate. If the change in flux is high enough, the magnet can float. This is the concept behind
mag-lev trains. This greatly reduces the effect of friction. Levitation can also be achieved with AC current and a solenoid.
This will cause the magnetic field to continuously change, creating an eddy current, which will cause a magnetic field 
to oppose this change, creating a repulsion.

\section{Inductance}
Consider a circuit with changing current. This will produce a changing magnetic field and an induced EMF.
This is known as \textbf{self-inductance}. The term "L self-inductance" refers to the proportionality constant $L$
in $\phi_{B}=LI$. So, $$\epsilon_{ind}=-\frac{d\phi_{B}}{dt}=-L\frac{dI}{dt}$$
In the case of a solenoid will length $l$, $N$ loops, loop-radius $r$, and current $I$. Previously, $B=\mu_{0}I\frac{N}{l}$.
So, $\phi_{B}=\pi r^{2}N\cdot \mu_{0}I\frac{N}{l}=LI$. Thus, $L=\mu_{0}\cdot \pi r^{2}\cdot \frac{N^{2}}{l}$. The SI unit for 
self-inductance is $\frac{\text{Volt sec}}{\text{A}}$ or a \textbf{Henry}. Every circuit has a finite, non-zero amount of self-inductance.

\subsection{RL Circuits}
Consider the circuit with battery of potential $V$, resistor of resistance $R$, and self-inductor of inductance $L$.
this is known as an \textbf{RL circuit}.
Here, the self-inductance fights against the increase in current.
We can {\it not} use Kirchoff's Loop Rule here because the magnetic flux is changing. So,
$$\oint_{C}\vec{E}\cdot d\vec{l}=-\frac{d\phi_{B}}{dt}=-L\frac{dI}{dt}=0+IR-V$$
This $0$ comes from there being no resistance in the self-inductor / no electric field. Rearranging,
$$L\frac{dI}{dt}+IR-V=0$$
Solving the differential equation, 
$$\frac{L}{V-IR}dI=dt\Rightarrow -\frac{L}{R}\cdot \ln(V-IR)=t+C$$
$$V-IR=\exp\left[-\frac{R}{L}\cdot (t+C)\right]=Ce^{-\frac{Rt}{L}}\Rightarrow
I=\frac{V}{R}\cdot (1-\frac{C}{V}\cdot e^{\frac{t}{L}})$$
At $t=0$, $I=0$. So, $C=V$. Thus,
$$I(t)=I_{max}\left(1-e^{-\frac{R}{L}t}\right)\text{ where } I_{max}=\frac{V}{R}$$
If this battery is shorted out, the voltage is now zero and the current is decreasing, but the self-inductance fights that change.
We solve a similar diff. eq. for this case,
$$L\frac{dI}{dt}+IR=0\Rightarrow -\frac{L}{IR}dI=dt$$
$$-\frac{L}{R}\ln(IR)=t+C\Rightarrow I=\frac{1}{R}\exp\left[-\frac{R}{L}\cdot (t+C)\right]$$
Using boundary conditions,
$$I(t)=I_{max}e^{-\frac{R}{L}t}$$
where $I_{max}$ is the current before the voltage is removed. Notice that the exponential term is the same in both cases.

\subsection{Magnetic Field Energy}
The battery was producing the heat in the resistor as the current increased, but when the battery is taken away,
the heat is still being produced; this energy is from the magnetic field. We can measure this energy by integrating the 
power / heat.
$$\int_{0}^{\infty}Pdt=I_{max}^{2}R\int_{0}^{\infty}e^{-\frac{2R}{L}t}dt=\frac{1}{2}LI_{max}^{2}$$
This energy is stored exclusively in the solenoid. Using the previous results of $B$ in terms of $I$ and $L$, we substitute to get
$$=\frac{1}{2}\cdot \mu_{0}\pi r^{2}\frac{N^{2}}{l}\cdot \left(\frac{Bl}{N\mu_{0}}\right)^{2}
=\frac{B^{2}l\pi r^{2}}{2\mu_{0}}=\frac{B^{2}}{2\mu_{0}}\cdot \pi r^{2}l
$$
Notice that $\pi r^{2}l$ is the volume of the solenoid where the magnetic field is (taking outside solenoid to be $0$). So,
the magnetic field energy density is
$$\rho_{B}=\frac{B^{2}}{2\mu_{0}} \frac{\text{J}}{\text{m}^{3}}$$
% QUESTION: in the light bulb experiment, Lewin #20 27:30, the bulb path with higher self-inductance takes longer to light.
% However, shouldn't that path also take slower to die down? why does it look like they turn off at the same rate?
% ANSWER: Heat is related to power which is propto I^2. So, though I does fall off as e^(-Rt/L), heat and then brightness 
% falls off by that to the power 2 which makes the time to die down almost unnoticeable.

\subsection{RL Circuit with AC}
Consider a RL circuit with alternating voltage $V=V_{0}\cos\omega t$. The differential equation now is
$$L\frac{dI}{dt}+IR-V_{0}\cos\omega t=0$$
This is pretty difficult. However, the solution ends up being,
$$I=\frac{V_0}{\sqrt{R^{2}+(\omega L)^{2}}}\cos(\omega t-\phi)\text{ where }\tan\phi=\frac{\omega L}{R}$$
This $\phi$ is the delay caused by the self inductance. Notice if $\omega=0$, \textbf{DC current}, 
the front term is simply $\frac{V_{0}}{R}$ which is expected. This term in the front is $I_{max}$.
Also notice that the $\omega L$ plays a "pseudo-resistance" from the self-inductance. In a radio, 
this can be used to cut-out higher frequencies using a self-inductor.

\subsection{Levitation}
It seems that in the levitation experiment, due to AC current, half the time there will be levitation and the other half
attraction. But, there is a net repulsion - this is due to self-inductance. So, the \textbf{phase lag} here 
is $\arctan \frac{\omega L}{R}$. Because of this delay, the net result is repulsion because the induced current and EMF are not in sync.

\section{Magnetic Materials}
Similar to how electric fields can induce dipoles or orient permanent dipoles,
magnetic fields can do the same and orient atoms that have magnetic dipole moment.
Lowering temperature and increasing field strength increases the ability for the field to do this.
A \textbf{magnetic dipole moment} is $\vec{u}=I\vec{A}$ for a loop with area $\vec{A}$ (normal) 
in closed loop with current $I$. By quantum mechanics, all materials will produce a magnetic dipole 
moment to some degree that opposes a permanent magnetic field.

\subsection{Diamagnetism}
By quantum mechanics, all materials, when they are exposed to a magnetic field, will oppose the field and generate an EMF that opposes the external field.
Note that this is not Lenz Law since the magnetic field doesn't have to change.

\subsection{Paramagnetism}
Some molecules and atoms have magnetic dipole moments themselves. If there is no external field, \textbf{vaccum field},
their orientation will be chaotically distributed. Otherwise, they will have a tendency to orient themselves (north pole)
in the direction of the field. Consider an atom (closed loop) next to a bar magnet (north). The net force, summing the Lorentz forces,
will be towards the north of the magnet. So, if a paramagnetic material is put in a non-uniform field, it will be pulled towards 
the strong side of the field. Oxygen is paramagentic but liquid oxygen is extremely more dense and has a lower temperature which means
it has a high $\kappa_{m}$ even though it is paramagnetic. So, in the presence of a strong non-uniform field, it can be lifted up.

\subsection{Feromagnetism}
In a feromagnetic material, the atoms have permanent magnetic dipole moments, but there are domains where the dipoles are 100\% aligned.
When a magnetic field is applied, the domains can flip. When it is taken away, some domains may stay while others may change due to thermal agitation.
This means that the material can become magnetic after being in the presence of a magnetic field (think paperclips).
Feromagnetism also has attraction to the strong side of the field, but the attraction is undoubtedly stronger.
At a certain temperature, the domains break apart - this is known as the \textbf{Curie temperature}.
Cobalt, Nickel, and Iron are the most common feromagnetic materials. Gadolinium is paramagnetic in the summer but feromagnetic in the winter.

\subsection{Barkhausen Effect}
If you connect a solenoid to an amplifier in a circuit and move a magnet fast towards the solenoid, current is induced producing a sound in the AMP.
If you put a feromagnetic material in the solenoid and move the magnet slowly, some domains begin to flip causing a change in magnetic flux 
which induces current and creates a crackling noise in the AMP. With this, you can hear the domains flipping.

\subsection{Relative Permeability}
For some materials, regardless of type of -magnetic, the magnetic field inside is proportional to the vaccum field.
$$\vec{B}=\kappa_{m}\vec{B}_{vac}$$
Here, $\kappa_{m}$ is the \textbf{relative permeability}. This is commonly expressed as $\kappa_{m}=1+\xi_{m}$.
Here, $\xi_{m}$ known as the \textbf{magnetic susceptibility}. 
\[
\begin{cases}
    \text{Diamagnetic} & \xi_{m}<0    \\
    \text{Paramagnetic} & \xi_{m}>0    \\
    \text{Feromagnetic} & \xi_{m}\gg 0 \\
\end{cases}
\]

\section{Maxwell's Equations}
\subsection{Magnetic Moment}
Consider calculating the magnetic moment of an atom of hydrogen. For the electron, we know $m_{e}$ and $e$.
Let $R$ be the Bohr radius of the atom. So, for the atom, $\mu=IA=I\pi R^{2}$. The Coloumb force of the electron 
to the nucleus (proton) is what provides its centirpetal acceleration. So,
$$\frac{e^{2}}{4\pi \epsilon_{0}R^{2}}
=\frac{mv^{2}}{R}\Rightarrow v=\sqrt{\frac{e^{2}}{4\pi \epsilon_{0}mR}}$$
To find the current, we need to find the period and then substitute.
$$T=\frac{2\pi R}{v}\Rightarrow I=\frac{e}{T}$$
For hydrogen, $\mu\approx 9.3\times 10^{-29}\text{Am}^{2}$ which is known as the \textbf{Bohr magneton}.
From quantum mechanics, $\mu$ for any electron is quantized by this amount and can even be $0$. This is because the charge 
itself is "spinning" around the electron. So, $\mu$ for the atom is the vectorial sum of all the dipole moments.
For most atoms, $\mu$ is either 1 or 2 Bohr magnetons.

\subsection{Hysteresis Curve}
Let $\vec{B}=\vec{B}_{vac}+\vec{B}'$. As seen previously,
if $\vec{B}'\propto\vec{B}_{vac}$, then, $\vec{B}'=\xi_{m}\vec{B}_{vac}$. This linearity holds for paramagentic material
but not for all feromagnetic materials which can acheive \textbf{saturation} where all dipoles are in one direction.
Lower temperatures increase the curve of $\vec{B}$ vs $\vec{B}'$. After the point of saturation, the only way to increase $\vec{B}$
is to increase $\vec{B}_{vac}$. Consider a solenoid with a feromagnet inside and plotting $B_{vac}$ and $B$.
When the current increases clockwise, $B_{vac}$ increases approaching saturation. When this happnes $B'$ still retains magnetic field
because of the domains. Making the current go back to $0$ decreases the field till $B_{vac}=0$ but $B'$ is still in one direction.
Then, reversing the direction of the current and increasing it causes $B_{vac}$ to "increase" (as negative) until $B'=0$ again.
Continuing causes this to reach saturation changing the direction of $B'$. Taking the current to $0$ causes $B'$ to remain but $B_{vac}$
to go to zero. Finally, increasing again, completes the loop on the graph. The shape formed is known as a \textbf{hysteresis curve}.
This is a closed curve, so given some $B_{vac}$, there are two possibilities for $B$ / $B'$ which means you can't tell the internal 
magnetic field of a feromagnet from the external field (you can't know the residual field from the magnet's "history").
This entire process can be achieved simply using AC current, though the originating \textbf{virginal curve} will be lost in visuals.
We can make the material \textbf{virginal} again by heating it above the Curie point, hitting it, or de-magnetizing it.

\subsection{Maxwell's Equations}
Maxwell's Equations are a set of four equations that govern the theory of electricity and magnetism.
We have Gauss' Law:
$$\oint \vec{E}\cdot d\vec{A}=\frac{Q_{free}}{\kappa \epsilon_{0}}$$
We have a result from the Biot-Savart Formulation due to the absence of magnetic monopoles:
$$\oint\vec{B}\cdot d\vec{A}=0$$
We have the Faraday-Lenz Law:
$$\oint \vec{E}\cdot d\vec{l}=-\frac{d\phi_{B}}{dt}$$
We have the modified Ampere's Law:
$$\oint \vec{B}\cdot d\vec{l}=\kappa_{m}\mu_{0}\left(I+\epsilon_{0}\kappa\frac{d\phi_{E}}{dt}\right)$$
All these laws work fine except for Ampere's Law which needs to account for $\kappa_{m}$
for magnetic materials.

\section{RC Circuits}
Consider a circuit with a battery, $V_{0}$, a capacitor $C$, a resistor $R$, and a switch completing the full loop. Since there are no magnetic fields,
going around the circuit,
$$V_{C}+IR-V_{0}=\oint \vec{E}\cdot d\vec{l}=0$$
Expressing in terms of time,
$$\frac{Q}{C}+R\frac{dQ}{dt}-V_{0}=0$$
Solving the differential equation,
$$Q=V_{0}C(1-e^{-\frac{t}{RC}});\, I=\frac{V_0}{R}e^{-\frac{t}{RC}};\,
V_{C}=V_{0}(1-e^{-\frac{t}{RC}})
$$
Now, if the switch moves so that it removes the battery from the loop after some time since $I\rightarrow 0$,
The charge will flow from the capacitor in the opposite direction. Our differential equation will now be
$$\frac{Q}{C}+R\frac{dQ}{dt}=0$$
If the switch switches at $t_{0}$,
$$I=-\frac{V_{0}}{R}e^{-\frac{t-t_{0}}{RC}}$$
Using this, the other variables can be found.

\subsection{Transformers}
Consider a coil with $N_{1}$ windings and self-inductance $L_{1}$ (primary) with a current $I_{1}$
running through it and extremely small $i_{1}$ powering a voltmeter. Consider a secondary coil wound 
so that there is a magnetic flux coupling between the coils. Let this have parameters $L_{2}$, $I_{2}$, $i_{2}$, $N_{2}$.
Using Faraday's Law on the primary coil, ignoring the mutual inductance between the two,
$$0-V_{1}=-L_{1}\frac{dI_{1}}{dt}=\varepsilon_{mid1}=-N_{1}\frac{d\phi_{B}}{dt}$$
For the other coil,
$$V_{2}+0=-L_{2}\frac{dI_{2}}{dt}=\varepsilon_{mid2}=-N_{2}\frac{d\phi_{B}}{dt}$$
So, $\left|\frac{V_{2}}{V_{1}}\right|=\frac{N_{2}}{N_{1}}$. Using AC current,
by manipulating the coil ratio, transformers are able to "transform" the voltage in a wire. This is helpful
since power lines need as high voltage as possible, so transformers "step-up" from the generator and "step-down"
to the consumer. If $R\ll \omega L$, there are no Eddy currents, and the magnetic flux is perfectly coupled,
then, the power delivered to the primary side is consumed on the secondary side. So, $I_{1}V_{1}=I_{2}V_{2}$ 
which gives the ratio of the currents (also related to coil ratio). In reality, even though the voltage is $110\sqrt{2}$,
volt meters only show the $110$.

\subsection{Induction Ovens}
Consider a copper ring with a slot and an iron nail filling the slot. Even though the ideal conditions are not met,
the current is high for high coil ratios. This high current means the power is extremely high and the nail will become 
hot and perhaps melt. This is how induction ovens work for cooking and welding.

\subsection{Spark Plugs}
Cars have coils that are run by the car battery (AC). To acheive these high currents, consider a similar circuit
but with a switch that opens/closes the whole circuit. When the switch is closed, if the winding ratio is high, there is
a high current in the secondary circuit. When the switch is opened, the resistance of the primary circuit increases 'infinitely'
and the time for the current to change is much smaller so $\frac{dI_{1}}{dt}$ and $\frac{d\phi_{B}}{dt}$ will be extremely large.
This creates a high induced EMF in the primary, but due to the coil ratio, the EMF in the secondary is now extremely high (kV scale).
This causes a spark on the spark plug (secondary) due to discharge (think breakdown electric field).

\section{LRC Circuits}
Consider a AC power supply with a capacitor, self-inductor, and resistor (clockwise in that order).
Using Faraday's Law,
$$V_{C}+0+IR-V_{0}\cos (wt)=-\frac{d\phi_{B}}{dt}=-L\frac{dI}{dt}$$
Substituting $I=\frac{dQ}{dt}$,
$$L\frac{d^{2}Q}{dt^{2}}+R\frac{dQ}{dt}+\frac{Q}{C}=V_{0}\cos(\omega t)$$
Solving the differential equation, 
$$I=\frac{V_{0}}{\sqrt{R^{2}+\left(\omega L-\frac{1}{\omega C}\right)^{2}}}\cos(\omega t-\phi)
\text{ where }\tan\phi=\frac{\omega L-\frac{1}{\omega C}}{R}$$
Here, $\xi=\omega L-\frac{1}{\omega C}$ is known as the \textbf{reactance} and 
$Z=\sqrt{R^{2}+\xi^{2}}$ is known as the \textbf{impedence}. The impedence can be thought of as an 
as a pseudo-resistance. This is a \textbf{steady-state solution} or something you get after a certain amount of time.
At the beginning, there are certain transient phenomenon that die out over time. As seen previously, 
the term in front $\frac{V_{0}}{Z}$ is $I_{max}$. The value of $Z$ that maximizes $I_{max}$
is known as the \textbf{resonance}. At resonance, $\xi=0$, so
$$\omega_{0}L-\frac{1}{\omega_{0}C}=0 \Rightarrow \omega_{0}=\frac{1}{\sqrt{LC}}$$
This $\omega_{0}$ has a subscript because it is associated with the \textbf{resonance frequency}.
At this point, $I_{max}=\frac{V_{0}}{R}$, $\phi=0$, and $Z=R$.

\subsection{Driving Frequency}
Let $L,R,C$ be fixed and consider manipulating $\omega$ from to change the value of $Z$.
As $\omega\rightarrow 0$, $Z\rightarrow \infty$ and $I_{max}\rightarrow 0$.
This occurs when the capacitor is fully charged and AC becomes DC and current can't flow anymore.
As $\omega\rightarrow\infty$, $Z\rightarrow\infty$ and $I_{max}\rightarrow 0$. 
This occurs when the self-inductor is "fighting" against the current. Graphing $I_{max}$ against $\omega$
is known as a \textbf{resonance curve} with a maximum of $\frac{V_0}{R}$ at $\omega_{0}$. At $\omega<\omega_0$,
capacitance is dominant, but for $\omega>\omega_0$, inductance is dominant. At some $I\approx 0.7I_{res}$,
the width of the curve is $\Delta \omega\approx\frac{R}{L}$. The \textbf{quality} of the resonance is 
defined as $Q=\frac{\omega_{0}}{\Delta \omega}=\frac{1}{R}\cdot \sqrt{\frac{L}{C}}$. The $0.7$ is associated
with the current at {\it half power} ($0.7^{2}\approx 50\%$). Notice that if you are off-resonance and come on resonance, 
the increase in current can break a resistor. The idea of a resonance frequency is applicable to mechanical objects like humans.

\subsection{Acheiving Resonance}
Consider a driven LRC circuit with a fixed frequency and $V_{0}$. The average power of a light-bulb 
connected to this is $\expect{I^{2}R}=\frac{1}{2}I_{max}^{2}R$. The $\frac{1}{2}$ comes from the time average of $\cos^{2}$.
Let $\omega$ (fixed), be lower than the resonance frequency. To remedy this, we can increase L or C to lower $\omega_{0}$.
To increase $L$, we can bring a feromagnetic material inside the self-inductor; this technique can be used to create a variable
self-inductance.

\subsection{Radios and TVs}
Radios and TVs work by tuning to some resonance frequency. The antenna receives many frequencies, and the user changes the capacitor
to change the frequency and tune in to a particular program.

\subsection{Metal Detectors}
Metal detectors are set to resonance and introducing metal change $L$ which moves the system off resonance, setting off an alarm.
It does so using two coils with different L, R, C, and I. There is a mutual inductance $M$ since the magnetic field produced by one 
coil affects the other and vice versa. This creates two differential equations that are coupled. Solving this is hard, but it ends up
producing two resonant frequencies where at least one depends strongly on $M$. When a metal is introduced, Eddy currents are formed in it 
which offsets the magnetic field and moves the system off the resonant frequency. The system has a very high quality and is thus extremely sensitive.
Metal detectors that are hand held use a similar idea but with two concentric coils.

\end{document}
 